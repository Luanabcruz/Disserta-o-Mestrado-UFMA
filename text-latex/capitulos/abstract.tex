\begin{resumo}[Abstract]
 \begin{otherlanguage*}{english}

Dry Eye Syndrome is one of the most frequently reported eye diseases in ophthalmologic practice. The diagnosis of this condition is a challenging task due to its multifactorial etiology. One of the most commonly used tests is the manual classification of tear film images captured with the Doane Interferometer or the Tearscope Plus. The instability of the tear film creates the need to develop computational techniques to support specialists in the diagnosis. This work presents a new approach for tear film classification based on texture analysis with phylogenetic diversity indexes and Ripley's K function. After feature extraction, we perform a Greedy Stepwise feature selection to determine the most representative samples. Finally, we use Support Vector Machine, Random Forest, Naive Bayes and Bayes Net to provide different classification approaches. This set of texture descriptors has enabled the proposed method to achieve promising results. The proposed method has achieved as best experimental results over 99\% of accuracy. This reveals that our method can be a practicable alternative to assist specialists diagnose the categories of the tear film interference patterns.

\textbf{Keywords}: Tear film lipid layer, Dry eye syndrome, Tear film images, Phylogenetic diversity index, Ripley's K-function
 \end{otherlanguage*}
\end{resumo}