\begin{resumo}

A Síndrome do Olho Seco é uma das doenças oculares mais frequentemente relatadas na prática oftalmológica. O diagnóstico dessa doença é uma tarefa desafiadora, devido à sua etiologia multifatorial. Um dos testes mais utilizados consiste na classificação manual das imagens do filme lacrimal capturadas com o Interferômetro Doane ou Tearscope Plus. A instabilidade do filme lacrimal cria a necessidade de desenvolver técnicas computacionais para apoiar especialistas no diagnóstico. Este trabalho apresenta uma nova abordagem para a classificação de imagens do filme lacrimal, baseada em análise de textura dos índices de diversidade filogenética e na função K de \textit{Ripley}. Após a extração de características, é realizada uma etapa de seleção de características que melhor discriminam as amostras utilizando o \textit{Greedy Stepwise}. E por fim, foram usados os algoritmos \textit{Support Vector Machine}, \textit{Random Forest}, \textit{Naive Bayes} e \textit{Bayes Net} para fornecer diferentes abordagens de classificação. O método proposto utilizando estes descritores de texturas revelou resultados promissores. Os melhores resultados experimentais obtiveram taxas de classificação superiores a 99\% de acerto. Isso revela que o método proposto pode ser uma alternativa viável para ajudar especialistas a diagnosticar as categorias dos padrões de interferência do filme lacrimal.

\textbf{Palavras-chave}: {Camada lipídica do filme lacrimal, Síndrome do olho seco, Imagens do filme lacrimal, Índice de diversidade filogenética, Função K de \textit{Ripley}}.
\end{resumo}